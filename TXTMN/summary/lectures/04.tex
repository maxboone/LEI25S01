\section{Text Categorization}

We can separate classification tasks in three distinct
categories: binary classification (yes/no), multi-class
classification (a/b/c) and multi-label
classification (nil/a/a,b/...).

\subsection{Task Definition}

\subsubsection{Text Unit}

First define the text unit, i.e. the size and boundary
of the document. For example, complete documents (articles,
emails), sections (minutes, speeches) or sentences (language
identification, sentiment classification).

\subsubsection{Categories}

What is the category that we want to extract from the documents,
i.e. spam/no spam, relevant/irrelevant, language, sentiment, stance,
topic/subtopic (i.e. which type of cancer), warning (i.e. detect hate
speech).

\subsubsection{Pre-processing}

We want to have features for documents, for example a vector
from each document with the same dimensionality. Using the bag
of words you would have a high-dimensional vector that is very
sparse, the advantage is that it is very transparent and you can
show from the weights why a classifier learned a specific model.
Alternatively, you can use embeddings which is less interpretable.

Using words as features on the vectors $x_i$ and the class label
as $y_i$. Before we use the words as features we tokenize them, and
generally we also lowercase and remove punctuation. BERT models
come cased which keeps capitalization and diacritical marks, and uncased
which removes these.
Further steps include, but are not limited to, removing stopwords,
lemmatizing or stemming, or adding phrases (i.e. not good)
as a single feature.

\subsubsection{Feature Selection}

Limiting the amount of features reduces the dimensionality and
avoids overfitting to the training data. Furthermore, due to the
long tail of the word distribution, we only want to use words that
appear in multiple documents.

Globally, we can fix the vocabulary size and keep only the top-n most
frequent terms. Rare terms can also be avoided by using a cut-off on
the term frequency. For example in the CountVectorizer you can set parameters
to have only terms that occur in a ratio of max\_df documents and at least
min\_df times.

\subsubsection{Term Weighting}

We can add the feature weights to the document-term matrix in binary
(occurs or not),
integer (term count) or a real value (more advanced). Real values can
often contain
more information than just the counts, a popular weighting scheme is
Tf-Idf. In this case
there are two components: the term frequency (tf) and inverse
document frequency (idf).

In Tf-Idf the term frequency (tf) counts how often the term occurs in
the document.
Instead of using the raw value, we use a 1 + log value such that the
term counts are
normalized closer to eachother. Meaning, 1 becomes 1, 10 becomes 2
and 100 becomes 3.

The inverse document frequency (idf) uses the intuition that the most
frequent terms
are not very informative. It gives the number of documents that the
term occurs in and
takes the log of the inverse of that.

\begin{definition}[Tf-Idf]
  \begin{align*}
    \text{tf} = 1 + \log(\text{term}_\text{count})
  \end{align*}
  \begin{align*}
    \text{idf} = \log(\frac{|\text{documents}|}{df_\text{term}})
  \end{align*}
  \begin{align*}
    \text{tf-idf} = \text{tf} \times \text{idf}
  \end{align*}
\end{definition}

\subsubsection{Classifier Learning}

If we use word features, we need an estimator that is well-suited for
high-dimensional and sparse data. Such as Naive Bayes, Support Vector Machines
or Random Forests. If we have embeddings, we can use transfer learning,
see lecture 6. Alternatively, we can use in-context learning with a
generative LLM.

\paragraph{Naive Bayes}

One of the older models is Naive Bayes which has been used for SPAM
classification
in emails for a long time. It uses the prior probability of each category in the
training data and use the posterior probability distribution over the
possible categories.

In essence, we take a document and calculate for all classes the
probability of that document
and that class, times the probability of the class in general,
divided by the probability of
the document. But, considering that the document probability is a
constant given, we can eliminate
that term and only compute the probability of the document and the
class, times the probability of
the class, and take the class where this is the highest.

We can calculate the probability of the document and the class by
taking the probability of each
term given the class. When we have each term's probability we can
multiply the probabilities and
get a single probability. This is also what makes this naive, as it
assumes that the terms are
independent.
However, this will fail, as we (almost) always have terms in the
document with the probability zero,
multiplying the entire probability to zero. This can be solved by
adding a smoothing function to the
terms. For example, laplace smoothing (add-one smoothing) assumes
that each term occurs one additional
time.

This approach does the two assumptions that the terms are independent
of eachother and assigns the same
probabilities to terms regardless of their position in the document.
This results in a correct estimation, but inaccurate probabilities,
as only the ordering
can be used to estimate the correct class.

\begin{definition}[Naive Bayes]
  ~\\
  \begin{align*}
    y = \operatorname*{argmax}_{k \in {1, ..., K}} p(C_k)
    \prod^n_{i=1} p(x_i | C_k)
  \end{align*}
  \\
  Where $x_i$ is the probability of the term in the class, meaning how often
  it occurs in this document, divided by how often it occurs in the class.
  Apply add-one (laplace) smoothing such that words that don't occur in the
  document don't make the probability zero.
\end{definition}

\subsubsection{Evaluation}

Split data into a test and training set, for example 80\% train and
20\% test. Don't train on the
test set to prevent overfitting on your dataset. For hyperparameter
tuning, use a validation set that
is part of your train set, generally using cross validation.

Accuracy is often not suitable as the classes are often unbalanced.
High accuracy in one class might mean
a low accuracy in the other. It depends on the task what we are
interesting in, for example you want a spam
filter to mark important emails as not spam more than that you want
to mark spam emails as spam.

Generally, precision and recall are used and evaluated on all
classes. Where precision measures how many of
the assigned labels are correct, and recall is how many of the true
labels were assigned.
The harmonic mean of the precision and recall is the F1 score which
combines the precision and recall.

\begin{definition}[Precision]
  \begin{align*}
    \text{Precision} = \frac{\text{True Positive}}{\text{True
    Positive} + \text{False Positive}}
  \end{align*}
\end{definition}

\begin{definition}[Recall]
  \begin{align*}
    \text{Recall} = \frac{\text{True Positive}}{\text{True Positive} +
    \text{False Negative}}
  \end{align*}
\end{definition}

\begin{definition}[F1-score]
  \begin{align*}
    \text{F1-score} = 2 \times \frac{\text{Precision} \times
    \text{Recall}}{\text{Precision} + \text{Recall}}
  \end{align*}
\end{definition}

\subsection{Quiz}

\begin{quiz}[The classification task for the 20 newsgroup text dataset is]
  ~\\
  Multi-class
\end{quiz}

\begin{quiz}[10.000 documents, shark occurs in 10 documents]
  ~\\
  idf is $log(10000 / 10) = 3$
\end{quiz}

\begin{quiz}[Document with length 100, in which the term shark occurs once]
  ~\\
  tf is 1 + log(1) = 1 (note that the document length is irrelevant for the tf)
\end{quiz}

\begin{quiz}[Why do we need laplace smoothing for Naive Bayes?]
  ~\\
  To prevent zero probabilities for classes
\end{quiz}

