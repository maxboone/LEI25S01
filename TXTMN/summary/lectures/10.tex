\section{Sentiment Analysis}

Sentiment analysis started as people started to produce opinionated
content in social media. For example, product reviews, comments or
posts from users are interesting to analyise. Example tasks are to
aggregate positive and/or negative opinions, compare aspects of products
from reviews, analyse sentiment to political issues, companies or products, or
to detect how opinions change over time.

The term sentiment analysis is too generic, always specify what the task at
hand really entails. Examples of this are opinion mining, sentiment
classification,
aspect-based sentiment extraction, subjectivity analysis, affect
analysis, emotion detection or
stance detection.

\begin{definition}[Stance Detection]
  Is someone in-favor, neutral or opposed towards a specific
  statement or topic.
\end{definition}

Depending on the task, the labels differ. Basic classes for sentiment
are negative,
postiive or neutral. Objective versus subjective. Emotion can be joy,
fear, anger, sadness,
disgust, shame or guilt. Stance can be pro, con or neutral.

Aside from basic classes such as negative, positive or neutral, we can
also use ordinal scales (such as product reviews) that have a numerical
range. In the case that we use orinal labels, we need a loss function
that works with numeric values. This means that we need to use a regression,
such as linear regression for continuous scales and ordinal regression if the
output is categorical (but ordered).

In summary, if we do document or sentence level sentiment analysis, we can
do classification or regression on the document or sentences. If we do entity
and aspect level sentiment analysis, we want to relate the sentiment to features
of a product, event or entity and use extraction and classification.

\subsection{Ordinal Regression}

We can not represent some ordinal scales as a continuous variable, even though
the values are ordered, there should not exist for example a negative or a half
value, or the distance between multiple ordered values might differ.

\begin{definition}[Loss function for Ordinal Regression]
  \begin{align*}
    P(y \leq j | \theta_j,w,X) = \frac{1}{1 + e^{-(\theta_j-wX)}}
  \end{align*}

  Where $y$ is the target variable, $\theta_j$ the threshold for class $j$,
  $X$ the input instances and $w$ the weights to be learned.
\end{definition}

In this case the model learns the thresholds and the weights in parallel
during training. The default training loss for transformer is the softmax,
which is suitable for classification problems. However, if you want to do
linear regression or ordinal regression, this loss function changes. For
example using a mean-squared error as loss function, or another loss function
that takes into account the ordinal nature of the labels.

\subsection{Aspect-based Sentiment Analysis (ABSA)}

In aspect-based sentiment analysis, we try to extract the sentiment that
someone expresses at some place and time about aspects of entities. For example,
someone might be postive about the travel times of public transport
but is negative
about the delays that often occur.

The target is to find quintuple(s) with the following values:

\begin{definition}[Aspect-based Sentiment Analysis Quintuple]
  ~
  \begin{itemize}
    \item \textbf{Entity} - the opinion target, entity, event or topic.
    \item \textbf{Aspect} - the aspect or feature of the entity.
    \item \textbf{Sentiment} - the sentiment score of the aspect.
    \item \textbf{Holder} - the opinion holder.
    \item \textbf{Context} - the time and location of the expression
  \end{itemize}
\end{definition}

An example of this in a review would be (iPad, Price, Reasonable,
Username, Date).
We can extract them as follows:

\paragraph{Entity}

In reviews, given by the metadata. In news or social media, it needs
to be extracted
from the text. NER or event detection.

\paragraph{Aspect}

Information Extraction and aspect categorization is needed. Aspects
are domain dependent
or even product dependent. They might be given in reviews as
metadata. Challenging as a
phrase like "the screen is dim" relates to the brightness.

\paragraph{Sentiment}

We are looking for sentiments of a given aspect, once we have the
entity and aspect, we
can classify the sentiment of the sentence(s) describing the aspect.

\paragraph{Opinion Holder}

In tweets and reviews, usually the author. In news, needs to be
extracted from the
text and NER can be used.

\paragraph{Context}

In tweets and reviews, date or location stamp. In news, we need time expression
recognition and/or geolocation classification.

\subsubsection{Knowledge Bases}

It helps to use a product database for this analysis. This helps to narrow down
which products exist (as someone might refer to a different product)
and it helps to know
which aspects a given product type has.

\subsubsection{Challenges}

A review can contain more than one product and references to that
product can span
over multiple sentences. Moreover, the same review can contain the
opinions of multiple
opinion holders. This can result into different aspects with
different sentiments and
co-references. Reviews can also contain factual statements (i.e. the
product was returned)
that state sentiment (negative) and sentiment words might not express
sentiment (i.e. i was looking for a bright screen).
Sentiment words in general are ambiguous, a bedside lamp that is
bright might be negative, but
a camera flash that is bright is positive.

\subsection{Quiz}

\begin{quiz}[What do you need besides sentiment classification model
  for customer reviews on an ordinal scale?]
  ~\\
  Labelled customer reviews for finetuning the model
  Change the loss function to ordinal regression
  GPU computing
\end{quiz}

\begin{quiz}[What are difference between sentiment classification and
  stance detection?]
  ~\\
  The labels can be different.
  Sentiment classification is about one item, stance detection is
  about the relation between two items.
\end{quiz}

\begin{quiz}[Write down a quintuple from a review]
  ~
  \begin{itemize}
    \item \textbf{Entity} LEGO Creator 3 in 1 Magical Unicorn Toy
    \item \textbf{Aspect} Value
    \item \textbf{Sentiment} Great
    \item \textbf{Holder} Bjorn
    \item \textbf{Context} US, 18th of October 2024
  \end{itemize}
\end{quiz}
