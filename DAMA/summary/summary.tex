\documentclass{article}

\usepackage[english]{babel}
\usepackage[a4paper, margin=2cm]{geometry}
\usepackage{float}
\usepackage{graphicx}
\usepackage[colorlinks=true, allcolors=blue]{hyperref}
\usepackage{amsmath}
\usepackage{amsthm}
\usepackage[many]{tcolorbox}
\usepackage{cmbright}
\usepackage{caption}
\usepackage{subcaption}
\usepackage{mathtools}

\theoremstyle{definition}
\newtheorem{definition}{Definition}[section]
\tcolorboxenvironment{definition}{
  colback=blue!5!white,
  boxrule=0pt,
  boxsep=1pt,
  left=2pt,right=2pt,top=2pt,bottom=2pt,
  oversize=2pt,
  sharp corners,
  before skip=\topsep,
  after skip=\topsep,
}

\theoremstyle{quiz}
\newtheorem{quiz}{Quiz}[section]
\tcolorboxenvironment{quiz}{
  colback=green!5!white,
  boxrule=0pt,
  boxsep=1pt,
  left=2pt,right=2pt,top=2pt,bottom=2pt,
  oversize=2pt,
  sharp corners,
  before skip=\topsep,
  after skip=\topsep,
}

\theoremstyle{note}
\newtheorem*{note}{Note}
\tcolorboxenvironment{note}{
  colback=yellow!5!white,
  boxrule=0pt,
  boxsep=1pt,
  left=2pt,right=2pt,top=2pt,bottom=2pt,
  oversize=2pt,
  sharp corners,
  before skip=\topsep,
  after skip=\topsep,
}

\DeclarePairedDelimiter\normv{\lVert}{\rVert}


\usepackage[T1]{fontenc}

\title{Advances in Data Mining - Summary}
\author{Max Boone}
\date{\today}

\begin{document}
\section{Introduction to Data Mining}

If we look at the change in processing speed and the
speed of data collection, we can deduce that every 3
years the processing speed increases by a factor 4 and
the collected data by a factor 8.

\begin{definition}[Moore's Law]
  Processing speed doubles every 18 months
\end{definition}

\begin{definition}[Kryder's Law]
  Hard disk capacity doubles every 12 months
\end{definition}

\begin{definition}[Lyman \& Varian]
  The amount of collected data doubles every year
\end{definition}

Meaning, every 3 years, the overflow of data (or flood)
is doubling. Besides the total amount of data, the shape
of the data has also changed over time.
This new data is also called Big Data, generally characterized
by the "Three Vs of Big Data" (IBM):

\begin{definition}[Three Vs of Big Data (IBM)]
  ~
  \begin{itemize}
    \item \textbf{Volume} - Large size of data (terabytes, petabytes).
    \item \textbf{Velocity} - Speed of added data is fast, and needs
      to be processed fast.
    \item \textbf{Variety} - Structured and unstructured data, such
      as text, sensor data, audio, video, click streams, log files and more.
  \end{itemize}
\end{definition}

To be able to process such big data, we need to do fuzzy queries and
we want to be able to run the processes in a streaming fashion. Traditionally
these are easy to execute, but complex in this context.
The trade-off therein is accuracy for speed (a rough answer in
real-time is sufficient).

\subsection{Examples}

\subsubsection{Recommender Systems}

Given petabytes of sales data in the form \textit{<user\_id, item\_id>}
we want to do recommendations to the \textit{current\_user\_id}
that just clicked or bought \textit{current\_item\_id} and
we have only milliseconds to do this.

This is a fairly simplified scenario, as you generally have
a lot more information on users to do these recommendations.

\subsubsection{Data Stream Mining}

In some situations, the incoming stream of data is too large
or too fast to be stored, and needs to be analyzed on-the-fly
with limited memory.

Examples of this are to detect fraud in electronic transactions, showing
the correct advertisements, prognostic health monitoring of a fighter jet,
detection of DoS attacks, or returning rankings from current interests
or click streams from users.

\subsection{Data Mining Paradigms}

We can group data mining activities into two separate paradigms: supervised
and unsupervised learning.

\begin{definition}[Supervised Learning]
  We have a set of defined inputs and defined outputs and
  try to build a model that converts input into these outputs
\end{definition}

\begin{definition}[Unsupervised Learning]
  The outputs are characteristics and generally tries to learn
  similarity from the data.
\end{definition}

\subsection{Overfitting}

If we would train our model on all the data that we have, we
have the risk to overfit on the data. This means that it follows
the data closely, but will not work with unseen data. Thus, we
separate the training data from test and evaluation data, and
use approaches to avoid overfitting on the training data.

For example, we can use $k$-Fold Cross-Validation where we split
the training data (stratified) in to $k$ folds and use ($k - 1$)
folds for training and $1$ for testing. Repeat this $k$ times
and average the results. A downside of this is that it is
computationally expensive (but parallelizable).


\section{Preprocessing}

All text needs a clean-up of some kind, either the documents might
have encoding errors, the text
might be scanned (through OCR) or recognized (from speech). Digital
input is also not always clean,
as it might contain things like layout, disclaimers, headers, tables,
or markup. We want to get rid
of this data before we use it in models, but it might contain useful
information (i.e. in XML or JSON)
that we can use.

\subsection{Encoding}

Text is encoded for storage, where characters are stored as numbers.
Classically this used ASCII but it
only encodes simple alphanumeric characters that are used in American
English. A universal and independent
standard for encoding is Unicode, and rendering of the characters is
implemented by the software. A popular
implementation of Unicode (UTF-8).

\subsection{Cleaning}

When we do a text mining task we generally have to start with the
creation of a dataset for our task.
This means that we likely have to digitize, convert and clean textual
content to run our mining task on
and can take the majority of the work of the task.

\subsubsection{Regular Expressions}

One way to get relevant content from textual data is to use regular
expressions, these are expressions
that can be used to search text. For example, a Dutch postal code is
\texttt{/[1-9][0-9]{3} [A-Z]{2}/}.
These expressions can be used to either extract information from
texts, or to remove terms or private
data from texts.

However, these might get too complex as the expressions can also start
hitting (parts) of words, you might need to account for capitalization
or different ways of writing a term.

\subsubsection{Spacy}

An alternative for Regular Expressions is to use Spacy, which converts words
in the text to tokens and then you can simply match on tokens (see
next section).

\subsection{Tokenization}

Text can be converted to tokens for comparison with oneanother, as you might
have different ways of writing the same word. For example,
\texttt{isn't it?} can
be written as \texttt{is not it?} and contains four tokens (three
words, one punctuation).

\begin{definition}[Token]
  An instance of a word or punctuation occurring in a document
\end{definition}

\begin{definition}[Word Type]
  A distinct word in the collection (without duplicates)
\end{definition}

\begin{definition}[Term]
  A token when used as a feature (or index), generally in normalized form.
\end{definition}

\begin{definition}[Token count]
  Number of words in a collection/document, includes duplicates.
\end{definition}

\begin{definition}[Vocabulary size]
  Number of unique terms (word types); the feature size or dimension
  size for bag-of-words.
\end{definition}

The choice of the terms (i.e. whether to use or not to use punctuation) depends
on the task that you are trying to complete. For example, in
forensics, you might
want to specifically look at the use of punctuation.

\subsubsection{Sentence Splitting}

Besides splitting on tokens, we might need to split text into
sentences. For example,
when we look at relations between multiple entities in the same
sentence. Or, if we want
to look at the sentiment of each sentence (such as in reviews). This
can be difficult due
to markup being used (bullet points), abbreviations, different uses
of punctuation or line
breaks in the document.

\subsubsection{Sub-word Tokenization}

Each new document will add new terms, that might relate to words that
we used earlier. For
example, the words model, models and modeling all have the same
"model" root. If we split these
words into subwords we can match on \textit{parts} of the words that
we have seen before.
One of the sub-word tokenizers is byte-pair encoding, which works as follows:

\begin{definition}[Byte-pair encoding (BPE)]
  Create a vocabulary of all characters. Merge the most occurring
  adjacent characters to form a new token
  and apply the new token. Repeat and create new subwords this way
  until we have $k$ novel tokens.
\end{definition}

\subsubsection{Lemmatization and Stemming}

We might want to normalize words that are written down differently to
the same term. For example, think, thinking and thought all have the same
meaning in a different tense. In this case you can take the \textbf{Lemma}
or the \textbf{Stem}.

\begin{definition}[Lemma]
  The dictionary form of a word, i.e. the infinitive for a verb
  (thought $\rightarrow$ think) and the singular for a noun (mice
  $\rightarrow$ mouse).
\end{definition}

\begin{definition}[Stem]
  The prefix of a word, i.e. computer, computing, computers, compute
  all share \texttt{comput}.
\end{definition}

We mostly prefer lemmas over stems, as they can merge more different
presentations
of the text to the same terms.

\subsection{Edit Distance}

For spelling correction and normalization we can look at the
"closest" correct word
to the word that was written. The metric that we can use to calculate
this is by Levenshtein distance, where insertion, deletion and substitution
all have a cost of 1. The match with the lowest cost in that case has
the lowest edit distance.
You can extend this cost function by for example lowering the costs of
characters that are next to eachother on a keyboard or are phonetically
similar.

\subsection{Quiz}

\begin{quiz}[What is optical character recognition (OCR)?]
  ~\\
  A technique for converting the image of a printed text to digital text.
\end{quiz}

\begin{quiz}[What is the main limitation of ASCII?]
  ~\\
  It can only encode letters that occur in the American English alphabet.
\end{quiz}

\begin{quiz}[What does the RegEx match?]
  ~\\
  Regular Expression: \verb|/<[^>]+>/| \\
  Any HTML/XML tag.
\end{quiz}

\begin{quiz}[Can you make a language-independent lemmatizer?]
  ~\\
  No, because a lemmatizer uses a dictionary.
\end{quiz}

\begin{quiz}[What is the levenshtein distance between shrimp and shrop?]
  ~\\
  2
\end{quiz}

\section{Vector Semantics}

\subsection{Vector Space Model}

In the bag of words approach, we can represent
documents and queries are in a vector space with
the words as dimensions. Each document can be
represented by a vector in this space.

The problem here is that the documents result in
very sparse vectors with a very high dimensionality.

The alternative is to not present documents as words
but rather by: topics or word embeddings. Topics are
discussed in lecture 9.

\subsection{Word Embeddings}

The basic idea behind representing words as
embeddings is that:

\begin{definition}[Distributional Hypothesis]
  The context of a word defines its meaning.
\end{definition}

For example, if we have \textit{a bottle of foobar},
\textit{a glass of foobar} and \textit{foobar gets you really drunk},
then it is likely that foobar is an alcoholic
beverage. Meaning, that a word can be assigned
a value by the words that surround it, and words
with similar values are likely similar words.

The idea is to create a dense vector space where we
place words that are similar close to eachother. The
dimensionality is between 100 to 400 here, which is
low dimension in NLP terms but rather high in other
disciplines. The similarity between the words is learned,
not from lemmas. The words are mapped to syntactically and semantically
similar words in a continuous dense vector space using the
distributional hypothesis.

These embeddings are generated through feed-forward neural
networks with the vector dimension as the number of output
nodes and the probabilities in the output are normalized using
the softmax function (which will exaggerate the differences
and give a clear choice from the vector).

\subsubsection{Word2Vec}

Word2Vec is an early efficient predictive model to learn the weights
for word embeddings. However superseded by BERT, it is still
used regularly as it is very efficient. It has a single hidden layer,
so it is not a deep neural network, and searches for the probability
of words are likely to show up near another word.

After training, the hidden layer then has the weights of each word
respective (completely connected) to the other words. We start with
the document, extract a vocabulary and try to represent it as a lower
dimensional vector.

This is a supervised task, but we did not label the data ourselves meaning
it is a self-supervised task (or approach).
Word2Vec does this by applying skip-gram with negative sampling, which
means that it:

\begin{enumerate}
  \item Take a target word, and a neighbouring context window as
    positive examples
  \item Randomly sample other words as negative samples
  \item Train a classifier (with one hidden layer) to distinguish
    these two cases
  \item The learned weights of the hidden layer are the embeddings
  \item Update previous embeddings of the word if necessary
\end{enumerate}

It scales well, learned embeddings can be re-used, and training can be done
incrementally. However, embeddings are static, meaning the same word always
has the same "meaning". Embeddings can be used as input for classifiers but
can't be updated while training the classifier.

\subsection{Document Embeddings}

It might also be useful to generate embeddings for documents, as to create
vectors from documents. For example, you could take the geometric average of
all words in the document, but many documents will end up close to the center
in this case.

An alternative is to use \texttt{doc2vec} where it generates a
separate embedding
for each paragraph in the document takes the words that it co-occurs
with as a positive
and words that it doesn't as negative. This can then be used as an
input to a classifier.

\subsection{Quiz}
\begin{quiz}[What is the role of the distributional hypothesis in
  training word embeddings?]
  ~\\
  Words that occur in similar contexts get similar representations
\end{quiz}


\section{Text Categorization}

We can separate classification tasks in three distinct
categories: binary classification (yes/no), multi-class
classification (a/b/c) and multi-label
classification (nil/a/a,b/...).

\subsection{Task Definition}

\paragraph{Text Unit}

First define the text unit, i.e. the size and boundary
of the document. For example, complete documents (articles,
emails), sections (minutes, speeches) or sentences (language
identification, sentiment classification).

\paragraph{Categories}

What is the category that we want to extract from the documents,
i.e. spam/no spam, relevant/irrelevant, language, sentiment, stance,
topic/subtopic (i.e. which type of cancer), warning (i.e. detect hate
speech).

\paragraph{Pre-processing}

We want to have features for documents, for example a vector
from each document with the same dimensionality. Using the bag
of words you would have a high-dimensional vector that is very
sparse, the advantage is that it is very transparent and you can
show from the weights why a classifier learned a specific model.
Alternatively, you can use embeddings which is less interpretable.



\section{Example Data}

In supervised learning, we need data that is labelled to train
our models. You can either use existing labelled data, or create
new labelled data.

\subsection{Existing Data}

We can use a benchmark dataset (i.e. from Kaggle and the sorts),
existing manually / human labelled data or labelled user-generated
content.

\subsubsection{Benchmark Data}

This uses labelled datasets that are made specifically to evaluate
and compare different methods and labels. Generally it doesn't matter
that much if the data is older for developing new methods, as long as it
is well-labelled and clean. Examples for text classification is the Reuters
Corpus Volume 1, for named entity recognition CoNLL-2003 which uses labelled
newspaper texts, for sentiment classification, IMDb movie reviews contains
50000 movie reviews that are positive or negative.

Advantages to using benchmark data is that it is high quality,
reusable, available
and you can compare results to other methods and approaches. However,
they are not
available for every specific problem and data type, and might not be
usable for your
task.

Another source for benchmarking datasets is Huggingface, however
there is not strong
curation for the datasets on Huggingface.

\subsubsection{Existing Labels}

Sources like papers, patents and other curated libraries have texts that are
annotated with labels by humans already. For example, patents have specific
classifications and categories assigned to them that show that it is relevant
to a given set of labels.

Advantages are that it is high-quality data, potentially large and
often freely available.
However, not available for every specific problem or data type and
might not be directly
suitable for training classifiers.

\subsubsection{Social Media (user-generated) content}

Social media posts can contain tags and metadata that label its
content. For example,
twitter has hashtags that can be used to detect sentiment or a
relevant topic to a
post. Another example is customer reviews to learn sentiment and
opinion, as the rating
generally aligns with the sentiment of the text.

Advantages are that it is available in many languages and
human-created. However, it can
be inconsistent, might be low quality and is an indirect signal (it
  was not intended as
a specific label).

\subsection{Create labelled data}

Sometimes, we don't have pre-existing labelled datasets that we
can use for training. In this case, we first take a sample of
the items (or documents) that we want to label. Second, we define
a set of categories that we want to label the data with. Third, we write
annotation guidelines to label the items with the categories. Finally,
test with annotators whether the instructions are clear and refine the
guidelines until they are. Ensure that the guidelines are clearly defined
but not too trivial.

With the instructions, you can use crowdsourcing (mTurk, Crowdflower) or
domain experts to annotate the texts. Compare the labels by different
annotators on the same text to estimate the reliability of the data
(inter-rater agreement).

Ensure that you have at at least dozens/hundreds per category, the
more the better
and the more difficult the problem, the more examples you need. If
you use crowdsourcing,
ensure that you have a check in the task, i.e. dummy questions, or
say the work is compared
to expert annotations.

\subsubsection{Quality Control}

Ensure that the same text is labelled by multiple persons, such that
you can measure
the agreement over labels between different annotators. This gives
the reliability of
the annotated data and can also be used as a measure for the
difficulty of the task.
A measure to use for that is Cohen's Kappa:

\begin{definition}[Cohen's Kappa]
  \begin{align*}
    \kappa = \frac{p(a) - p(e)}{1 - p(e)}
  \end{align*}

  Where $p(a)$ is the agreed percentage, and $pr(e)$ the expected agreement
  based on the occurrence of each of the values.
\end{definition}


\section{Neural Models for Sequential Data}

When we talk about predicting the next word in a sequence,
in a classical sense we generally look at n-gram language models.
Here, given a sequence of tokens, we estimate the probability
distribution over the vocabulary for the next token.

In neural networks, we can use the word embeddings of the previous
words to predict the next word. This works better to generalize ``unseen''
data. For example, in the sentence ``I have to make sure that the cat
gets fed.'',
we want to predict: ``I forgot to make sure that the dog gets ...''.

When we would use n-grams, we never saw ``gets fed'' after the word dog,
but we have seen the word cat which is semantically and syntactically similar
to cat and likely has a near embedding.

\subsection{Sequential Text}

Language is sequential data, and feed-forward neural networks do not
model such temporal aspects and models the words independently from eachother.
Extension of the feed-forward neural network for modelling sequential data is
a recurrent neural network (RNN). This adds the weights calculated in the hidden
layer in the previous time stamp. This ensures that the past context is used in
calculating the output for the current input.

An example where we want to use these temporal neural networks is
part-of-speech (POS)
tagging, which labels (sequence labelling) words with their type
(noun, verb). For example, the word fish
is a noun in \textit{the fish swims} but a verb in \textit{we like to fish}.

A problem with using RNNs is that they only carry one time step
forward in the weights.
Although all previous steps are incorporated in the product of the
previous weights, the relationship
between words more distante is not modeled.

\subsubsection{Long Short-Term Memory (LSTM)}

A solution to this problem is a more complex RNN that takes a longer
context into account.
They have a separate vector retaining the context information that is
relevant for a longer
part of the sequence. This is updated in each step and carries
temporally relevant data.

A problem with this is that they are slow to train, as their
computation can not be parallelized.
Furthermore, even though they have a longer context, they can not
model relations that span multiple
sentences or paragraphs.

\subsection{Transformer Models}

A solution to these memory problems and word embeddings are transformer models.
This is basically a successor to the RNN that was generally used for
the encoding
and decoding of sequences but does not work with contexts well.

A transformer is a neural network with an added mechanism of ``self-attention''.
The intuition is that while embeddings of words are learned, attention is paid
to surrounding tokens and this information is integrated. In every
representation
that you get, it has some information about the surrounding tokens,
and compares the
relations of the token to all other tokens in the set (in parallel).

This makes transformer models faster than BiLSTMs and other RNNs to train.
It can run in parallel because the tokens and its relations can be modeled
independently.

The transformer model has an encoder-decoder architecture. In this context this
means that it consists of an input encoder, transformer blocks and a language
modeling head.

\subsubsection{Input Encoding}

The input encoder processes input tokens into a vector representation
for each token,
and includes with that the position of the input token in the
sequence. The output of
this encoder is information on what the word syntactically and
semantically means and
where it exists in the context.

\subsubsection{Transformer blocks}

Then, these input embeddings are fed into a multilayer network that
also includes all
previous embeddings in the sequence. It results in a set of vectors
$H_i$ that are embeddings
that include the learned context of the word. These output vectors
can be used as the basis for
other learning tasks or for generating token sequences.

Each input embedding is compared to all other input embeddings using
the dot product to calculate
a score. The larger the value, the more similar it is to the vectors
that are being compared. This
is computationally heavy and therefore the input for transformers is
maximized to a given number.

In summary, each input embedding plays three distinct roles in the
self-attention mechanism:

\begin{itemize}
  \item \textbf{The query:} Represents the current input and is used
    to compare against all other inputs to evaluate their relevance.
  \item \textbf{The key:} Represents the role or importance of each
    previous input in relation to the current input.
  \item \textbf{The value:} Provides the actual contextual
    information, weighted by the similarity score computed between
    the query and the key.
\end{itemize}

In essence, the query and key determine how much attention each input
should pay to others, while the value carries the actual content
being passed along. Each token in the input sequence is compared to
all other tokens, and these comparisons, represented as a weighted
sum, capture the contextual importance of other tokens relative to
the current token.

In summary, each input embedding plays three roles in the
self-attention mechanism:

\subsubsection{Language modeling head}

Finally, the trained / learned embeddings are passed through a final
transformer block and through
softmax over the vocabulary to generate a single (predicted) token.

\subsection{Applying transformer models}

Given a training corpus of text, we can train the transformer model
to predict the next token in a sequence. The goal here is to learn the
representations of meaning for words.

Then, using autoregressive generation, we incrementally generate words
by repeatedly sampling the next word based on the previous words. Finally,
teacher forcing forces the system to use the target tokens from training
as the next input $x_{t+1}$ instead of the decoder output $y_t$.

\subsection{BERT}

Using the transformer model as a base, we use pre-training to model a
language. But, instead of going only from left-to-right, we use both sides
as context when predicting words. Finally, instead of using decoding to
generate the text, we stick to only encoding and use only the output embeddings.
We can then use the output embeddings to do supervised learning.

This allows us to do masked language modelling. In this case, we randomly mask
words during training and try to predict what these words should be. A special
token is used as a boundary to separate sentences and allows to learn
the relation
between full sentences.

\subsubsection{Fine-tuning}

The output of an initial train from BERT can be used to fine-tune the
model which
is less computationally expensive. We take the network learned by the
pretrained model
and add a neural net classifier on top of it with supervised data, to
perform a specific
downstream task (such as named entity recognition). This basically
replaces the head
in the transformer model with a task-specific model. Using a
pre-trained model to
fine-tune is also called transfer learning.

\begin{definition}[Transfer Learning]
  Using a pre-trained model (such as BERT) and fine-tune the
  parameters using labeled
  data from downstream tasks (supervised learning).
\end{definition}

For classification tasks, the input of each text is given a special
token \texttt{CLS},
the output vector in the final layer for the \texttt{CLS} input
represents the entire input
sequence and can be used as input to a classifier head.

For sequence-pair classification, or to find the relation between two sentences,
a second special token \texttt{SEP} is used to separate the two input sequences.
The model processes both sequences jointly, with the \texttt{CLS} token
capturing the combined representation of the pair, which is then used
by the classification head to determine the relationship.

Finally, for named entity recognition, we model the head to give the
label for each input token. Each token's output representation from the
final layer is passed through a classification layer to predict
its corresponding entity label, enabling token-level classification.

If we use a pre-trained model without fine-tuning, this is called zero-shot.
This can also be used for models that were fine-tuned by someone else
on a different
task such as using sentence similarity for ontology mapping,
newspaper benchmark on
tweets or a different language.

\begin{definition}[Zero-shot learning]
  Using a pre-trained model without fine-tuning, or a previously
  fine-tuned model (on a different task).
\end{definition}

\begin{definition}[Few-shot learning]
  Fine-tuning a pre-trained model on a small sample size.
\end{definition}

Although this works very well, the due to the complexity of the
transformer model
it is difficult to explain why the outputs from the model are what
they are. Sometimes,
word models are still used to better trace back why a model does
certain predictions.

\subsection{Quiz}

\begin{quiz}[What is the kind of task used to learn word embeddings]
  ~\\
  Language modelling
\end{quiz}

\begin{quiz}[Which statements about context in sequential models are true?]
  ~\\
  LSTMs are sequential models with longer memories than traditional RNNs. \\
  BERT models compute the relation between each pair of tokens in the input. \\
  The attention mechanism in Transformer models has quadratic
  complexity relative to the input length. \\
  The maximum input length for BERT models is limited by
  computational memory. \\
\end{quiz}

\begin{quiz}[What is the meaning of teacher forcing]
  ~\\
  Using true tokens instead of predicted tokens in generative training.
\end{quiz}

\begin{quiz}[Consider a sentiment analysis task, what do we need to
  build a prediction model using transfer learning?]
  ~\\
  GPU computing, a pre-trained Chinese BERT model, a regression
  layer, and the 1000 items for supervised fine-tuning.
\end{quiz}

\section{Generative LLMs}

Generative pre-trained transformers are decoder-only transformers.
Given a prompt, they can generate output text. When these transformers
becomer larger in the number of parameters, they are trained for large amounts
of data and fine-tuned for conversational use.

The first model that was sufficiently large to be practically useful
beyond language
modelling was GPT-3. If we use a pre-trained or previously fine-tuned
model without
fine-tuning, that is called zero-shot use.

In few-shot learning, we can give a few (3-5) examples and fine-tune the
model with those examples. In LLMs, these examples are given in the prompt
and the actual model is not changed.

The idea that the models get new capabilities that are not present in
smaller models is called emergent abilities. These can not be predicted
simply by extrapolating the performance of smaller models.

\begin{definition}[Emergent Abilities]
  ~\\
  Abilities that a model gets by scaling it up, that was not present
  in smaller models.
\end{definition}

Due to these abilities, we can actually ``fine-tune'' large models in-context
by providing instructions and adding examples in the prompt given to the model.
This does not change the model itself, but rather the model learns to
do something
inside the context of the given prompt. This requires fewer examples,
the model is not
updated and what is learnt is of temporary nature.

Generally, if you have the labelled data, fine-tuning a (BERT) model
is the better
option. You could also use an LLM to label data to use in fine-tuning
a BERT model.

\subsection{Output sampling}

In GPT models, we sample a word in the output from the transformer's
softmax distribution
that results with the prompt as the left context. We then use the
word embeddings
for the sampled word as additional left context, and sample the next word in the
same fashion. We continue generation until we reach an end-of-message
marker, or a
fixed length limit is reached.

Because we sample from a probability distribution, the output is
probablistic and
not deterministic in nature. We can do this sampling in different
ways. Greedy decoding
samples the most probable token at each step, which might be locally
optimal but not
globally.

\begin{definition}[Greedy decoding]
  Choose the most probably token from the softmax distribution at each step.
\end{definition}

If we choose globally, we can take the highest possible outcome from different
choices at each token. This is computationally heavy as we have to multiply all
the possible choices with the successive choices.

As an alternative, we can sample from the top-k most probable tokens.
When $k = 1$ the
sampling is identical to greedy decoding. If we set $k > 1$ it will
sometimes select
a word that is probably enough, generating a more diverse text. We
can also use top-p sampling,
which samples from words above a cut-off probability, and dynamically
increases and decreases
the pool of word candidates.

\begin{definition}[Top-k sampling]
  Sample a word from the top $k$ most probable words.
\end{definition}

\begin{definition}[Top-p sampling]
  Sample a word from words with a probability above $p$.
\end{definition}

Finally, we can use temperature based sampling, which changes the
probability distribution smoothing
the probability based on for example smoothing in low temperatures. A
higher temperature will result in
more randomness and diversity, a lower temperature produces a  more
focused and deterministic output.

\begin{definition}[Temperature sampling]
  Reshape the probability distribution instead of truncating it.
\end{definition}

\subsection{Pre-training LLMs}

The pre-training task's objective in LLMs is to get as close as
possible to predict the next
word, using cross-entropy as a loss function. It measures the
difference between a predicted
probability distribution for the next word compared to the true
probability distribution:

\begin{definition}[Cross-entropy loss]
  \begin{align*}
    L_\text{CE} = - \sum_{w \in V} y_t [w] \log(\hat{y}_t) [w]
  \end{align*}

  Considering there is only one correct next word, it is one-hot, and
  the formula can be adapted:

  \begin{align*}
    L_\text{CE}(\hat{y}_t, y_t) = -\log(\hat{y}_t) [w_{t+1}]
  \end{align*}
\end{definition}

\subsubsection{Pre-training Data}

Uses \textbf{Common Crawl} web data, a snapshot of the entire crawled
web. Wikipedia and books.
Data is filtered for quality, for example sites with PII or adult
content. Boilerplate text is removed and
the data is deduplicated. Finally, data is also filtered for safety,
such as through the detection of illegal and toxic texts.

\subsection{Evaluating Generative Models}

If we want to evaluate generative models, we can generally compare it for
tasks such as summarization or question answerings to human output.
We can do this
by measuring word overlap or semantic overlap.

It is also convenient to have a quantitative measure for the quality of
a pretrained model that is not task-specific. We can measure how well the
model predicts unseen text, i.e. by feeding it a text that it has not seen
before and see whether it completes the text correctly.

This is also called the perplexity of the model on an unseen test
set. In essence, this measures how suprised the model is to see the text.
It is defined as the inverse probability that the model assigns to the test set,
defined for model $\theta$:

\begin{definition}[Perplexity]
  \begin{align*}
    \text{Perplexity}_\theta(w_{1:n}) = P_\theta(w_{1:n})^{-\frac{1}{n}}
  \end{align*}
\end{definition}

\subsection{Finetuning LLMs}

When we fine-tune BERT models, we update the network to the supervised
task by applying transfer learning. For GPT-2 this was also possible as
it did not have too much parameters to do this feasibly (BERT around
110M, GPT-2 around 137M).
Fine-tuning LLMs in this way is computationally not feasible, given
GPT-3 has around 175B params
and LLaMA3 8B or 70B params.

Instead, we can do the following:

\subsubsection{Continued pretraining}

We take the parameters and retrain the model on new data, using the same
method of word prediction and the loss function as was done for retraining.

\subsubsection{Parameter-efficient finetuning (PEFT)}

We only (re-)train a subset of the parameters on new data. An example of this is
LoRa: Low-Rank (dimensionality reduction) adaptation of Large
Language Models. In this case, the pretrained
model weights are frozen, and we inject trainable rank decomposition matrices
into each layer of the transformer. This can reduce the number of
trainable parameters
by a 10000 times. This results into somewhat of a proxy model of the
original model.

\subsubsection{Supervised finetuning (SFT)}

We take a small LLM with around 2 or 3 billion paramters and train
it to produce exactly the sequence of tokens in the desired output.

\subsubsection{Reinforcement learning from human feedback (RLHF)}

Let humans indicate which output they prefer, then train the model
on this distinction. This is used often in tasks where it is difficult to define
a ground truth, but where humans can easily judge the quality of the
generated output.

In the first step we learn a reward model from the pairwise comparisons done
by humans. In the second step, we fine-tune the language model to the
learned reward
model, which aligns the model with human preferences.

\subsection{Conversational LLMs}

Conversational LLMs are GPT models that are fine-tuned for conversational
usage. This is done through supervised finetuning, by training it
with conversational
data on the web (i.e. from Reddit). Then, RLHF was used to fine-tune
the model to give
appropriate and desired responses.

\subsection{Potential Harms}

Due to the probablistic nature of generative language models, hallucinations
occur. The content that is generated is plausible and fluent, but does not have
to be necessarily correct.

GPT-3.5 contains unfaithful information in 25\% of the summaries. Both ChatGPT
and GPT-4 are very poor at yes/no question answering (with less than
50\% accuracy).
Only 20\% of the responses in healthcare of ChatGPT and GPT-3.5 were agreed with
by experts.

LLMs also contain biases from the learned text, this can be a problem
when these models are used to provide assesments or suggestions to humans. For
example in the checking of resumes, the models will have a bias in
selected candidates.

Using web content results in problems due to the uncertainty of copyright
on scraped pages. The owner of a website might not want their content
to be used in
training. And personal information is available on the web and should
not be returned
by LLMs.

Finally, there are ethical concerns with the training and usage of
these models as
they consume a large amount of energy in training and require
high-end GPUs. A single
run in ChatGPT costs 1000x more energy than one search in Google.
When crowdsourcing
for RLHF and filtering tasks, cheap labour is generally used to
prevent harmful and
toxic content generation.

\subsection{Quiz}

\begin{quiz}[Why does greedy decoding not always lead to the best output?]
  ~\\
  Because the local choices are not necessarily the global optimum.
\end{quiz}

\begin{quiz}[What is the difference between fine-tuning and
  in-context learning?]
  ~\\
  Fine-tuning uses more training data
  In-context learning uses instructions, fine-tuning does not
  With fine-tuning the model is updated, with in-context learning it is not
\end{quiz}

\begin{quiz}[What are emergent abilities of LLMs]
  ~\\
  Abilities that are not present in smaller models but only in larger models
\end{quiz}

\begin{quiz}[Can we use LLMs for annotation]
  ~\\
  It depends on the task
\end{quiz}

\section{Boosting}

Given a collection of learners in the decision tree, consider
a class of weak learners $h_m$ which is only slightly better
than random guessing. These are generally leafs of the decision
tree and are also called ``decision stumps''.

Build a sequence of classifiers $h_1, h_2, ..., h_k$:
\begin{itemize}
  \item $h_1$ is trained on the original data
  \item $h_2$ assigns more weight to misclassified cases by $h_1$
  \item $h_i$ assigns more weight to misclassified cases by $h_1, ..., h_{i-1}$
\end{itemize}

The final classifier is then an ensemble of $h1,...,h_k$ with
weights $\alpha_m$ per classifier $h_m$:
$G(x) = \text{sign}(\sum a_m h_m(x))$.

\begin{definition}[Learner Weights]
  The weight $\alpha_m$ for a learner is determined by its performance,
  quantified by the error rate $\epsilon_m$. Higher-performing learners
  receive greater positive weights, while poor-performing learners may
  receive negative weights. The weight is computed as:
  \begin{align*}
    \alpha_m = \frac{1}{2} \ln \left( \frac{1 - \epsilon_m}{\epsilon_m} \right),
  \end{align*}
  where $\epsilon_m$ is the error rate of learner $m$.
\end{definition}

\begin{definition}[Data Weights]
  On each iteration $G_m$ to $G_{m+1}$, we increase the weights
  for $G_{m+1}$ based on the $G_m$'s misclassified cases. We increase
  the weight by a factor $\exp(\alpha_m)$.
\end{definition}

\subsection{AdaBoost}

AdaBoost is an implementation of this idea, and works as follows:

\begin{definition}[AdaBoost]
  ~
  \begin{itemize}
    \item Initialize the observation weights $w_i = \frac{1}{N}$
    \item Loop over the classifiers $M$ as $m$
      \begin{itemize}
        \item Fit a classifier $h_m(x)$ to the training data using weights $w_i$
        \item Compute the weighted error $\epsilon_m = \frac{
            \sum w_i \times I(y_i \neq h_m(x_i))
          }{
            \sum w_i
          }$
        \item Compute $\alpha_m = \frac{1}{2} \ln ( \frac{1 -
          \epsilon_m}{\epsilon_m} )$
        \item For $i = 1$ to $N$
          \begin{itemize}
            \item If $x_i$ is misclassified, set data weight $w_i$ to
              $w_i \times \exp(\alpha_m)$
          \end{itemize}
      \end{itemize}
    \item Output $G(x) = \text{sign}(\sum \alpha_m h_m(x))$
  \end{itemize}
\end{definition}

\subsection{Gradient Boosting}

An alternative to AdaBoost is Gradient Boosting that has a bit more
theory behind the way how each iteration works. It uses a cost function
$L(u, F(x))$ which is the MSE in regression and log loss in the
classification. This loss function is differentiable and we can
use gradient descent here.

\begin{definition}[Gradient Boosting]
  ~
  \begin{itemize}
    \item Initialize the model with a constant value $F_0$, e.g. the mean
      of target values.
    \item For $m = 1$ to $M$
      \begin{itemize}
        \item Train weak learner $h_m(x)$ to minimize residuals from the current
          prediction using the loss function (c.q. train the model to
            the negative
          gradients).
        \item New prediction is $F_{m+1}(x) = F_m(x) + v \times h_m(x)$, where
          $v$ is the learning rate (e.g. 0.1)
      \end{itemize}
    \item Output $F_M(x)$ as the final model
  \end{itemize}
\end{definition}

\subsubsection{Loss Functions}

By default, we look at the MSE, but there are other loss functions to use for
the gradient descent as well, with the square loss repeated:

\begin{definition}[Square Loss]
  \begin{displaymath}
    L(y, F) = \frac{1}{2} (y - F)^2
  \end{displaymath}
\end{definition}

\begin{definition}[Absolute Loss]
  \begin{displaymath}
    L(y, F) = \left| y - F \right|
  \end{displaymath}
\end{definition}

\begin{definition}[Huber Loss]
  \begin{displaymath} L(y, F) =
    \begin{cases}
      \frac{1}{2} (y - F)^2 & \text{if } |y - F| \leq
      \delta \\
      \delta \left( |y - F| - \frac{1}{2} \delta \right) &
      \text{if } |y - F| > \delta
    \end{cases}
  \end{displaymath}

  Where $\delta$ refers to a parameter that decreases how
  extreme the output should be with respect to higher
  outliers.
\end{definition}


\section{Text Mining in Practice}

There are two forms of summarization, extractive summarization that is
composed completely of material from the source and abstractive summarization
that contains material not originally in the sources, but shorter paraphrases.

\begin{definition}[Extractive summarization]
  Completely composed of material from the source.
\end{definition}

\begin{definition}[Abstractive Summarization]
  Contains material not originally in the source, but shorter paraphrases
\end{definition}

An example of extractive summarization is a search engine snippet,
that literally
comes from the document but is not altered. Abstractive summarization
generates a
completely new summarization from the source material.

Extractive summarization is generally a supervised task, where we
classify each sentence
for inclusion or exclusion from the summary. Alternatively, we can
assign a score to
each sentence and take the top-k (or top-p) scores. It is very
reliable, as the information
given is sure to be from the original text, but it can be limited in fluency.

Abstractive summarization can use sequence-to-sequence models (such
as T5) to learn a mapping
between an input sequence and an output sequence. This can be trained
on pairs of
longer and shorter texts. Alternatively, we can use instruction-tuned
LLMs where we
provide a few examples for in-context learning. It provides a more
natural and fluent
result but there is a risk of hallucinations.

\subsection{Healthcare TM}

There is a large amount of data in healthcare which brings advantages
and challenges
for text mining tasks. Examples are demographic data, discharge
letters, imaging, diagnosis codes,
lab results or patient experience questionnaires. Aging population,
more chronic diseases and
innovation are drivers for applying text mining in healthcare.

\subsection{AutoScriber}

The idea behind autoscriber is to automate (part of) the tasks related
to medical annotation and documentation tasks from patient consultations.
It works by recording clinical conversations, generating a transcript from 
the conversation and feeding this text data into NLP tasks for entity recognition,
classification and summarization.

\subsubsection{Summarization}

As this research was done in 2021, the idea was to start with the automatic
transcript, preprocess it to remove stop words among others, and use tf-idf 
to extract keywords. These keywords are then filtered to extract relevant
keywords. Next, neural networks were applied (CNN-LSTM), and needed a lot 
more annotation.

One of the challenges is that the datasets are incomplete, they are collected
for a specific reason by a specific doctor, making it subjective and biased
data. Especially as the conversation is guided by the doctor, the contents will
change based on the suspicions and ideas from the doctor. For example, a cardiologist
will ask very different questions than a psychiatrist.
Another challenge was the correctness of the data, typos, abbreviations (that are ambiguous)
and missing metadata.

Finally, for summarization LLMs were employed as it is particularly necessary to
use fluent language. Extractive summarization will only contain parts of the conversation
and as there are multiple persons speaking this might give difficult to understand sentences
and results. Advantages are that no annotated data is necessary, and only examples are sufficient.
Moreover, low quality data is not a large issue here. However, it is difficult to evaluate the
outcomes of the LLMs and there is a risk of hallucinations.



\end{document}
