\documentclass{article}

\usepackage[english]{babel}
\usepackage[a4paper, margin=2cm]{geometry}
\usepackage{float}
\usepackage{graphicx}
\usepackage[colorlinks=true, allcolors=blue]{hyperref}
\usepackage{amsmath}
\usepackage{amsthm}
\usepackage[many]{tcolorbox}
\usepackage{cmbright}
\usepackage{caption}
\usepackage{subcaption}
\usepackage{mathtools}

\theoremstyle{definition}
\newtheorem{definition}{Definition}[section]
\tcolorboxenvironment{definition}{
  colback=blue!5!white,
  boxrule=0pt,
  boxsep=1pt,
  left=2pt,right=2pt,top=2pt,bottom=2pt,
  oversize=2pt,
  sharp corners,
  before skip=\topsep,
  after skip=\topsep,
}

\theoremstyle{quiz}
\newtheorem{quiz}{Quiz}[section]
\tcolorboxenvironment{quiz}{
  colback=green!5!white,
  boxrule=0pt,
  boxsep=1pt,
  left=2pt,right=2pt,top=2pt,bottom=2pt,
  oversize=2pt,
  sharp corners,
  before skip=\topsep,
  after skip=\topsep,
}

\theoremstyle{note}
\newtheorem*{note}{Note}
\tcolorboxenvironment{note}{
  colback=yellow!5!white,
  boxrule=0pt,
  boxsep=1pt,
  left=2pt,right=2pt,top=2pt,bottom=2pt,
  oversize=2pt,
  sharp corners,
  before skip=\topsep,
  after skip=\topsep,
}

\DeclarePairedDelimiter\normv{\lVert}{\rVert}


\usepackage[T1]{fontenc}

\title{Advances in Data Mining - Summary}
\author{Max Boone}
\date{\today}

\begin{document}
\section{Introduction to Data Mining}

If we look at the change in processing speed and the
speed of data collection, we can deduce that every 3
years the processing speed increases by a factor 4 and
the collected data by a factor 8.

\begin{definition}[Moore's Law]
  Processing speed doubles every 18 months
\end{definition}

\begin{definition}[Kryder's Law]
  Hard disk capacity doubles every 12 months
\end{definition}

\begin{definition}[Lyman \& Varian]
  The amount of collected data doubles every year
\end{definition}

Meaning, every 3 years, the overflow of data (or flood)
is doubling. Besides the total amount of data, the shape
of the data has also changed over time.
This new data is also called Big Data, generally characterized
by the "Three Vs of Big Data" (IBM):

\begin{definition}[Three Vs of Big Data (IBM)]
  ~
  \begin{itemize}
    \item \textbf{Volume} - Large size of data (terabytes, petabytes).
    \item \textbf{Velocity} - Speed of added data is fast, and needs
      to be processed fast.
    \item \textbf{Variety} - Structured and unstructured data, such
      as text, sensor data, audio, video, click streams, log files and more.
  \end{itemize}
\end{definition}

To be able to process such big data, we need to do fuzzy queries and
we want to be able to run the processes in a streaming fashion. Traditionally
these are easy to execute, but complex in this context.
The trade-off therein is accuracy for speed (a rough answer in
real-time is sufficient).

\subsection{Examples}

\subsubsection{Recommender Systems}

Given petabytes of sales data in the form \textit{<user\_id, item\_id>}
we want to do recommendations to the \textit{current\_user\_id}
that just clicked or bought \textit{current\_item\_id} and
we have only milliseconds to do this.

This is a fairly simplified scenario, as you generally have
a lot more information on users to do these recommendations.

\subsubsection{Data Stream Mining}

In some situations, the incoming stream of data is too large
or too fast to be stored, and needs to be analyzed on-the-fly
with limited memory.

Examples of this are to detect fraud in electronic transactions, showing
the correct advertisements, prognostic health monitoring of a fighter jet,
detection of DoS attacks, or returning rankings from current interests
or click streams from users.

\subsection{Data Mining Paradigms}

We can group data mining activities into two separate paradigms: supervised
and unsupervised learning.

\begin{definition}[Supervised Learning]
  We have a set of defined inputs and defined outputs and
  try to build a model that converts input into these outputs
\end{definition}

\begin{definition}[Unsupervised Learning]
  The outputs are characteristics and generally tries to learn
  similarity from the data.
\end{definition}

\subsection{Overfitting}

If we would train our model on all the data that we have, we
have the risk to overfit on the data. This means that it follows
the data closely, but will not work with unseen data. Thus, we
separate the training data from test and evaluation data, and
use approaches to avoid overfitting on the training data.

For example, we can use $k$-Fold Cross-Validation where we split
the training data (stratified) in to $k$ folds and use ($k - 1$)
folds for training and $1$ for testing. Repeat this $k$ times
and average the results. A downside of this is that it is
computationally expensive (but parallelizable).


\section{Preprocessing}

All text needs a clean-up of some kind, either the documents might
have encoding errors, the text
might be scanned (through OCR) or recognized (from speech). Digital
input is also not always clean,
as it might contain things like layout, disclaimers, headers, tables,
or markup. We want to get rid
of this data before we use it in models, but it might contain useful
information (i.e. in XML or JSON)
that we can use.

\subsection{Encoding}

Text is encoded for storage, where characters are stored as numbers.
Classically this used ASCII but it
only encodes simple alphanumeric characters that are used in American
English. A universal and independent
standard for encoding is Unicode, and rendering of the characters is
implemented by the software. A popular
implementation of Unicode (UTF-8).

\subsection{Cleaning}

When we do a text mining task we generally have to start with the
creation of a dataset for our task.
This means that we likely have to digitize, convert and clean textual
content to run our mining task on
and can take the majority of the work of the task.

\subsubsection{Regular Expressions}

One way to get relevant content from textual data is to use regular
expressions, these are expressions
that can be used to search text. For example, a Dutch postal code is
\texttt{/[1-9][0-9]{3} [A-Z]{2}/}.
These expressions can be used to either extract information from
texts, or to remove terms or private
data from texts.

However, these might get too complex as the expressions can also start
hitting (parts) of words, you might need to account for capitalization
or different ways of writing a term.

\subsubsection{Spacy}

An alternative for Regular Expressions is to use Spacy, which converts words
in the text to tokens and then you can simply match on tokens (see
next section).

\subsection{Tokenization}

Text can be converted to tokens for comparison with oneanother, as you might
have different ways of writing the same word. For example,
\texttt{isn't it?} can
be written as \texttt{is not it?} and contains four tokens (three
words, one punctuation).

\begin{definition}[Token]
  An instance of a word or punctuation occurring in a document
\end{definition}

\begin{definition}[Word Type]
  A distinct word in the collection (without duplicates)
\end{definition}

\begin{definition}[Term]
  A token when used as a feature (or index), generally in normalized form.
\end{definition}

\begin{definition}[Token count]
  Number of words in a collection/document, includes duplicates.
\end{definition}

\begin{definition}[Vocabulary size]
  Number of unique terms (word types); the feature size or dimension
  size for bag-of-words.
\end{definition}

The choice of the terms (i.e. whether to use or not to use punctuation) depends
on the task that you are trying to complete. For example, in
forensics, you might
want to specifically look at the use of punctuation.

\subsubsection{Sentence Splitting}

Besides splitting on tokens, we might need to split text into
sentences. For example,
when we look at relations between multiple entities in the same
sentence. Or, if we want
to look at the sentiment of each sentence (such as in reviews). This
can be difficult due
to markup being used (bullet points), abbreviations, different uses
of punctuation or line
breaks in the document.

\subsubsection{Sub-word Tokenization}

Each new document will add new terms, that might relate to words that
we used earlier. For
example, the words model, models and modeling all have the same
"model" root. If we split these
words into subwords we can match on \textit{parts} of the words that
we have seen before.
One of the sub-word tokenizers is byte-pair encoding, which works as follows:

\begin{definition}[Byte-pair encoding (BPE)]
  Create a vocabulary of all characters. Merge the most occurring
  adjacent characters to form a new token
  and apply the new token. Repeat and create new subwords this way
  until we have $k$ novel tokens.
\end{definition}

\subsubsection{Lemmatization and Stemming}

We might want to normalize words that are written down differently to
the same term. For example, think, thinking and thought all have the same
meaning in a different tense. In this case you can take the \textbf{Lemma}
or the \textbf{Stem}.

\begin{definition}[Lemma]
  The dictionary form of a word, i.e. the infinitive for a verb
  (thought $\rightarrow$ think) and the singular for a noun (mice
  $\rightarrow$ mouse).
\end{definition}

\begin{definition}[Stem]
  The prefix of a word, i.e. computer, computing, computers, compute
  all share \texttt{comput}.
\end{definition}

We mostly prefer lemmas over stems, as they can merge more different
presentations
of the text to the same terms.

\subsection{Edit Distance}

For spelling correction and normalization we can look at the
"closest" correct word
to the word that was written. The metric that we can use to calculate
this is by Levenshtein distance, where insertion, deletion and substitution
all have a cost of 1. The match with the lowest cost in that case has
the lowest edit distance.
You can extend this cost function by for example lowering the costs of
characters that are next to eachother on a keyboard or are phonetically
similar.

\subsection{Quiz}

\begin{quiz}[What is optical character recognition (OCR)?]
  ~\\
  A technique for converting the image of a printed text to digital text.
\end{quiz}

\begin{quiz}[What is the main limitation of ASCII?]
  ~\\
  It can only encode letters that occur in the American English alphabet.
\end{quiz}

\begin{quiz}[What does the RegEx match?]
  ~\\
  Regular Expression: \verb|/<[^>]+>/| \\
  Any HTML/XML tag.
\end{quiz}

\begin{quiz}[Can you make a language-independent lemmatizer?]
  ~\\
  No, because a lemmatizer uses a dictionary.
\end{quiz}

\begin{quiz}[What is the levenshtein distance between shrimp and shrop?]
  ~\\
  2
\end{quiz}

\section{Vector Semantics}

\subsection{Vector Space Model}

In the bag of words approach, we can represent
documents and queries are in a vector space with
the words as dimensions. Each document can be
represented by a vector in this space.

The problem here is that the documents result in
very sparse vectors with a very high dimensionality.

The alternative is to not present documents as words
but rather by: topics or word embeddings. Topics are
discussed in lecture 9.

\subsection{Word Embeddings}

The basic idea behind representing words as
embeddings is that:

\begin{definition}[Distributional Hypothesis]
  The context of a word defines its meaning.
\end{definition}

For example, if we have \textit{a bottle of foobar},
\textit{a glass of foobar} and \textit{foobar gets you really drunk},
then it is likely that foobar is an alcoholic
beverage. Meaning, that a word can be assigned
a value by the words that surround it, and words
with similar values are likely similar words.

The idea is to create a dense vector space where we
place words that are similar close to eachother. The
dimensionality is between 100 to 400 here, which is
low dimension in NLP terms but rather high in other
disciplines. The similarity between the words is learned,
not from lemmas. The words are mapped to syntactically and semantically
similar words in a continuous dense vector space using the
distributional hypothesis.





\section{Text Categorization}

We can separate classification tasks in three distinct
categories: binary classification (yes/no), multi-class
classification (a/b/c) and multi-label
classification (nil/a/a,b/...).

\subsection{Task Definition}

\paragraph{Text Unit}

First define the text unit, i.e. the size and boundary
of the document. For example, complete documents (articles,
emails), sections (minutes, speeches) or sentences (language
identification, sentiment classification).

\paragraph{Categories}

What is the category that we want to extract from the documents,
i.e. spam/no spam, relevant/irrelevant, language, sentiment, stance,
topic/subtopic (i.e. which type of cancer), warning (i.e. detect hate
speech).



\section{Example Data}

In supervised learning, we need data that is labelled to train
our models. You can either use existing labelled data, or create
new labelled data.

\subsection{Existing Data}

We can use a benchmark dataset (i.e. from Kaggle and the sorts),
existing manually / human labelled data or labelled user-generated
content.

\subsubsection{Benchmark Data}

This uses labelled datasets that are made specifically to evaluate
and compare different methods and labels. Generally it doesn't matter
that much if the data is older for developing new methods, as long as it
is well-labelled and clean. Examples for text classification is the Reuters
Corpus Volume 1, for named entity recognition CoNLL-2003 which uses labelled
newspaper texts, for sentiment classification, IMDb movie reviews contains
50000 movie reviews that are positive or negative.

Advantages to using benchmark data is that it is high quality,
reusable, available
and you can compare results to other methods and approaches. However,
they are not
available for every specific problem and data type, and might not be
usable for your
task.

Another source for benchmarking datasets is Huggingface, however
there is not strong
curation for the datasets on Huggingface.

\subsubsection{Existing Labels}

Sources like papers, patents and other curated libraries have texts that are
annotated with labels by humans already. For example, patents have specific
classifications and categories assigned to them that show that it is relevant
to a given set of labels.

Advantages are that it is high-quality data, potentially large and
often freely available.
However, not available for every specific problem or data type and
might not be directly
suitable for training classifiers.

\subsubsection{Social Media (user-generated) content}

Social media posts can contain tags and metadata that label its
content. For example,
twitter has hashtags that can be used to detect sentiment or a
relevant topic to a
post. Another example is customer reviews to learn sentiment and
opinion, as the rating
generally aligns with the sentiment of the text.

Advantages are that it is available in many languages and
human-created. However, it can
be inconsistent, might be low quality and is an indirect signal (it
  was not intended as
a specific label).

\subsection{Create labelled data}

Sometimes, we don't have pre-existing labelled datasets that we
can use for training. In this case, we first take a sample of
the items (or documents) that we want to label. Second, we define
a set of categories that we want to label the data with. Third, we write
annotation guidelines to label the items with the categories. Finally,
test with annotators whether the instructions are clear and refine the
guidelines until they are. Ensure that the guidelines are clearly defined
but not too trivial.

With the instructions, you can use crowdsourcing (mTurk, Crowdflower) or
domain experts to annotate the texts. Compare the labels by different
annotators on the same text to estimate the reliability of the data
(inter-rater agreement).

Ensure that you have at at least dozens/hundreds per category, the
more the better
and the more difficult the problem, the more examples you need. If
you use crowdsourcing,
ensure that you have a check in the task, i.e. dummy questions, or
say the work is compared
to expert annotations.

\subsubsection{Quality Control}

Ensure that the same text is labelled by multiple persons, such that
you can measure
the agreement over labels between different annotators. This gives
the reliability of
the annotated data and can also be used as a measure for the
difficulty of the task.
A measure to use for that is Cohen's Kappa:

\begin{definition}[Cohen's Kappa]
  \begin{align*}
    \kappa = \frac{p(a) - p(e)}{1 - p(e)}
  \end{align*}

  Where $p(a)$ is the agreed percentage, and $pr(e)$ the expected agreement
  based on the occurrence of each of the values.
\end{definition}

\subsection{Quiz}

\begin{quiz}[Why should we have multiple human annotators?]
  ~\\
  \begin{itemize}
    \item Because we need to estimate the reliability of the data.
    \item Because we need to measure the inter-rated agreement
      between annotators
    \item Because there is human interpretation involved in annotation
  \end{itemize}
\end{quiz}

\begin{quiz}[What is the interpretation of Kappa = 0]
  ~\\
  Measured agreement is equal to random agreement, negative means
  that there is more
  disagreement than random.
\end{quiz}


\section{Neural Models for Sequential Data}

When we talk about predicting the next word in a sequence,
in a classical sense we generally look at n-gram language models.
Here, given a sequence of tokens, we estimate the probability
distribution over the vocabulary for the next token.

In neural networks, we can use the word embeddings of the previous
words to predict the next word. This works better to generalize ``unseen''
data. For example, in the sentence ``I have to make sure that the cat
gets fed.'',
we want to predict: ``I forgot to make sure that the dog gets ...''.

When we would use n-grams, we never saw ``gets fed'' after the word dog,
but we have seen the word cat which is semantically and syntactically similar
to cat and likely has a near embedding.

\subsection{Sequential Text}

Language is sequential data, and feed-forward neural networks do not
model such temporal aspects and models the words independently from eachother.
Extension of the feed-forward neural network for modelling sequential data is
a recurrent neural network (RNN). This adds the weights calculated in the hidden
layer in the previous time stamp. This ensures that the past context is used in
calculating the output for the current input.

An example where we want to use these temporal neural networks is
part-of-speech (POS)
tagging, which labels (sequence labelling) words with their type
(noun, verb). For example, the word fish
is a noun in \textit{the fish swims} but a verb in \textit{we like to fish}.

A problem with using RNNs is that they only carry one time step
forward in the weights.
Although all previous steps are incorporated in the product of the
previous weights, the relationship
between words more distante is not modeled.

\subsubsection{Long Short-Term Memory (LSTM)}

A solution to this problem is a more complex RNN that takes a longer
context into account.
They have a separate vector retaining the context information that is
relevant for a longer
part of the sequence. This is updated in each step and carries
temporally relevant data.

A problem with this is that they are slow to train, as their
computation can not be parallelized.
Furthermore, even though they have a longer context, they can not
model relations that span multiple
sentences or paragraphs.

\subsection{Transformer Models}

A solution to these memory problems and word embeddings are transformer models.
This is basically a successor to the RNN that was generally used for
the encoding
and decoding of sequences but does not work with contexts well.

A transformer is a neural network with an added mechanism of ``self-attention''.
The intuition is that while embeddings of words are learned, attention is paid
to surrounding tokens and this information is integrated. In every
representation
that you get, it has some information about the surrounding tokens,
and compares the
relations of the token to all other tokens in the set (in parallel).

This makes transformer models faster than BiLSTMs and other RNNs to train.
It can run in parallel because the tokens and its relations can be modeled
independently.

The transformer model has an encoder-decoder architecture. In this context this
means that it consists of an input encoder, transformer blocks and a language
modeling head.

\subsubsection{Input Encoding}

The input encoder processes input tokens into a vector representation
for each token,
and includes with that the position of the input token in the
sequence. The output of
this encoder is information on what the word syntactically and
semantically means and
where it exists in the context.

\subsubsection{Transformer blocks}

Then, these input embeddings are fed into a multilayer network that
also includes all
previous embeddings in the sequence. It results in a set of vectors
$H_i$ that are embeddings
that include the learned context of the word. These output vectors
can be used as the basis for
other learning tasks or for generating token sequences.

Each input embedding is compared to all other input embeddings using
the dot product to calculate
a score. The larger the value, the more similar it is to the vectors
that are being compared. This
is computationally heavy and therefore the input for transformers is
maximized to a given number.

In summary, each input embedding plays three distinct roles in the
self-attention mechanism:

\begin{itemize}
  \item \textbf{The query:} Represents the current input and is used
    to compare against all other inputs to evaluate their relevance.
  \item \textbf{The key:} Represents the role or importance of each
    previous input in relation to the current input.
  \item \textbf{The value:} Provides the actual contextual
    information, weighted by the similarity score computed between
    the query and the key.
\end{itemize}

In essence, the query and key determine how much attention each input
should pay to others, while the value carries the actual content
being passed along. Each token in the input sequence is compared to
all other tokens, and these comparisons, represented as a weighted
sum, capture the contextual importance of other tokens relative to
the current token.

In summary, each input embedding plays three roles in the
self-attention mechanism:

\subsubsection{Language modeling head}

Finally, the trained / learned embeddings are passed through a final
transformer block and through
softmax over the vocabulary to generate a single (predicted) token.

\subsection{Applying transformer models}

Given a training corpus of text, we can train the transformer model
to predict the next token in a sequence. The goal here is to learn the
representations of meaning for words.

Then, using autoregressive generation, we incrementally generate words
by repeatedly sampling the next word based on the previous words. Finally,
teacher forcing forces the system to use the target tokens from training
as the next input $x_{t+1}$ instead of the decoder output $y_t$.

\subsection{BERT}

Using the transformer model as a base, we use pre-training to model a
language. But, instead of going only from left-to-right, we use both sides
as context when predicting words. Finally, instead of using decoding to
generate the text, we stick to only encoding and use only the output embeddings.
We can then use the output embeddings to do supervised learning.

This allows us to do masked language modelling. In this case, we randomly mask
words during training and try to predict what these words should be. A special
token is used as a boundary to separate sentences and allows to learn
the relation
between full sentences.

\subsubsection{Fine-tuning}

The output of an initial train from BERT can be used to fine-tune the
model which
is less computationally expensive. We take the network learned by the
pretrained model
and add a neural net classifier on top of it with supervised data, to
perform a specific
downstream task (such as named entity recognition). This basically
replaces the head
in the transformer model with a task-specific model. Using a
pre-trained model to
fine-tune is also called transfer learning.

\begin{definition}[Transfer Learning]
  Using a pre-trained model (such as BERT) and fine-tune the
  parameters using labeled
  data from downstream tasks (supervised learning).
\end{definition}

For classification tasks, the input of each text is given a special
token \texttt{CLS},
the output vector in the final layer for the \texttt{CLS} input
represents the entire input
sequence and can be used as input to a classifier head.

For sequence-pair classification, or to find the relation between two sentences,
a second special token \texttt{SEP} is used to separate the two input sequences.
The model processes both sequences jointly, with the \texttt{CLS} token
capturing the combined representation of the pair, which is then used
by the classification head to determine the relationship.

Finally, for named entity recognition, we model the head to give the
label for each input token. Each token's output representation from the
final layer is passed through a classification layer to predict
its corresponding entity label, enabling token-level classification.

If we use a pre-trained model without fine-tuning, this is called zero-shot.
This can also be used for models that were fine-tuned by someone else
on a different
task such as using sentence similarity for ontology mapping,
newspaper benchmark on
tweets or a different language.

\begin{definition}[Zero-shot learning]
  Using a pre-trained model without fine-tuning, or a previously
  fine-tuned model (on a different task).
\end{definition}

\begin{definition}[Few-shot learning]
  Fine-tuning a pre-trained model on a small sample size.
\end{definition}

Although this works very well, the due to the complexity of the
transformer model
it is difficult to explain why the outputs from the model are what
they are. Sometimes,
word models are still used to better trace back why a model does
certain predictions.

\subsection{Quiz}

\begin{quiz}[What is the kind of task used to learn word embeddings]
  ~\\
  Language modelling
\end{quiz}

\begin{quiz}[Which statements about context in sequential models are true?]
  ~\\
  LSTMs are sequential models with longer memories than traditional RNNs. \\
  BERT models compute the relation between each pair of tokens in the input. \\
  The attention mechanism in Transformer models has quadratic
  complexity relative to the input length. \\
  The maximum input length for BERT models is limited by
  computational memory. \\
\end{quiz}

\begin{quiz}[What is the meaning of teacher forcing]
  ~\\
  Using true tokens instead of predicted tokens in generative training.
\end{quiz}

\begin{quiz}[Consider a sentiment analysis task, what do we need to
  build a prediction model using transfer learning?]
  ~\\
  GPU computing, a pre-trained Chinese BERT model, a regression
  layer, and the 1000 items for supervised fine-tuning.
\end{quiz}

\section{Random Forests and Ensembles}

Random Forests are supervised learning models that can be used for
both classification and regression tasks. They are based on an
ensemble of decision trees, which individually split the dataset into
subsets by evaluating feature values. These splits occur at multiple
levels, enabling the model to progressively refine its predictions
for the target variable.

\subsection{Ensembles}

An ensemble in supervised learning is a collection of models that
works by aggregating the individual predictions. Generally, it is more
accurate than the base model. Regression generally averages the indivdual
predictions, and classification uses a majority vote.

It helps if there is more diversity between models, which can be achieved by
using randomization or multiple types of classifier models.

\subsubsection{Bagging}

Bootstrap Aggregating, in short Bagging, is an early implementation of this
idea. Here each tree is bootstrapped with random samples with replacement
from the original dataset.

\begin{definition}[Bagging]
  \begin{itemize}
    \item Take random samples with replacement
    \item Given a training set $D$ of size $n$, generate $m$
      new training sets $D_1,...,D_m$ each of size $n$.
    \item Replacement means that some observations will be
      repeated in each sample.
    \item For a large $n$, $D_i$ will have a fraction of $(1 -
      \frac{1}{e})$ samples.
      Meaning, around $63.2\%$ unique samples and $36/8\%$ duplicates.
  \end{itemize}
\end{definition}

Overfitting is avoided in this case as the learnes have little
correlation as they learn from different datasets. The optimal
amount of learners can be determined by cross-validation or OOB 
estimation.

\subsubsection{Random Subspace Method}

A follow-up of bagging is the random subspace method. Instead of
sampling, we build each tree in the ensemble from a random subset of
the attributes. This mainly works for high-dimensional problems and
individual trees won't over-focus on attributes that appear the most
predictive in the training set.

\subsubsection{Random Forests}

Finally, random forests first select a set of random attributes and
set the best attributes from the random subset on each node. The amount
of attributes is $\sqrt{p}$ for classification and $\frac{p}{3}$ attributes
for regression. It also uses out-of-bag estimation:

  \begin{definition}[Out of Bag Estimation]
    
  \end{definition}

\section{Boosting}

Given a collection of learners in the decision tree, consider
a class of weak learners $h_m$ which is only slightly better
than random guessing. These are generally leafs of the decision
tree and are also called ``decision stumps''.

Build a sequence of classifiers $h_1, h_2, ..., h_k$:
\begin{itemize}
  \item $h_1$ is trained on the original data
  \item $h_2$ assigns more weight to misclassified cases by $h_1$
  \item $h_i$ assigns more weight to misclassified cases by $h_1, ..., h_{i-1}$
\end{itemize}

The final classifier is then an ensemble of $h1,...,h_k$ with
weights $\alpha_m$ per classifier $h_m$:
$G(x) = \text{sign}(\sum a_m h_m(x))$.

\begin{definition}[Learner Weights]
  The weight $\alpha_m$ for a learner is determined by its performance,
  quantified by the error rate $\epsilon_m$. Higher-performing learners
  receive greater positive weights, while poor-performing learners may
  receive negative weights. The weight is computed as:
  \begin{align*}
    \alpha_m = \frac{1}{2} \ln \left( \frac{1 - \epsilon_m}{\epsilon_m} \right),
  \end{align*}
  where $\epsilon_m$ is the error rate of learner $m$.
\end{definition}

\begin{definition}[Data Weights]
  On each iteration $G_m$ to $G_{m+1}$, we increase the weights
  for $G_{m+1}$ based on the $G_m$'s misclassified cases. We increase
  the weight by a factor $\exp(\alpha_m)$.
\end{definition}

\subsection{AdaBoost}

AdaBoost is an implementation of this idea, and works as follows:

\begin{definition}[AdaBoost]
  ~
  \begin{itemize}
    \item Initialize the observation weights $w_i = \frac{1}{N}$
    \item Loop over the classifiers $M$ as $m$
      \begin{itemize}
        \item Fit a classifier $h_m(x)$ to the training data using weights $w_i$
        \item Compute the weighted error $\epsilon_m = \frac{
            \sum w_i \times I(y_i \neq h_m(x_i))
          }{
            \sum w_i
          }$
        \item Compute $\alpha_m = \frac{1}{2} \ln ( \frac{1 -
          \epsilon_m}{\epsilon_m} )$
        \item For $i = 1$ to $N$
          \begin{itemize}
            \item If $x_i$ is misclassified, set data weight $w_i$ to
              $w_i \times \exp(\alpha_m)$
          \end{itemize}
      \end{itemize}
    \item Output $G(x) = \text{sign}(\sum \alpha_m h_m(x))$
  \end{itemize}
\end{definition}

\subsection{Gradient Boosting}

An alternative to AdaBoost is Gradient Boosting that has a bit more
theory behind the way how each iteration works. It uses a cost function
$L(u, F(x))$ which is the MSE in regression and log loss in the
classification. This loss function is differentiable and we can
use gradient descent here.

\begin{definition}[Gradient Boosting]
  ~
  \begin{itemize}
    \item Initialize the model with a constant value $F_0$, e.g. the mean
      of target values.
    \item For $m = 1$ to $M$
      \begin{itemize}
        \item Train weak learner $h_m(x)$ to minimize residuals from the current
          prediction using the loss function (c.q. train the model to
            the negative
          gradients).
        \item New prediction is $F_{m+1}(x) = F_m(x) + v \times h_m(x)$, where
          $v$ is the learning rate (e.g. 0.1)
      \end{itemize}
    \item Output $F_M(x)$ as the final model
  \end{itemize}
\end{definition}

\subsubsection{Loss Functions}

By default, we look at the MSE, but there are other loss functions to use for
the gradient descent as well, with the square loss repeated:

\begin{definition}[Square Loss]
  \begin{displaymath}
    L(y, F) = \frac{1}{2} (y - F)^2
  \end{displaymath}
\end{definition}

\begin{definition}[Absolute Loss]
  \begin{displaymath}
    L(y, F) = \left| y - F \right|
  \end{displaymath}
\end{definition}

\begin{definition}[Huber Loss]
  \begin{displaymath} L(y, F) =
    \begin{cases}
      \frac{1}{2} (y - F)^2 & \text{if } |y - F| \leq
      \delta \\
      \delta \left( |y - F| - \frac{1}{2} \delta \right) &
      \text{if } |y - F| > \delta
    \end{cases}
  \end{displaymath}

  Where $\delta$ refers to a parameter that decreases how
  extreme the output should be with respect to higher
  outliers.
\end{definition}


\section{Support Vector Machines}

The basic intuition is that we want to separate a space
into multiple vectors that represent the center of a cluster
in that space. Then, if we draw a linear model that is orthogonal
to the line between both centers we can use that as a separator.
This works well in convex, non-verlapping distributions of equal
density. The center that is above the line is $\mu^+$ and below
the line is $\mu^-$.

\begin{definition}[Basic Linear Classifier]
  \begin{align*}
    w \times x &= t \\
    w &= \mu^+ - \mu^- \\
    t &= \frac{1}{2} \normv{\mu^+}^2 + \frac{1}{2} \normv{\mu^-}^2
  \end{align*}
\end{definition}

\section{Anomaly Detection}

There are many different use cases for anomaly detection,
and different approaches fit to different types of anomalies
that need to be detected. One such example is finding anomalies
in health care data, specifically about declarations, payments,
and corrections. Insurance companies can't check all the cases
individually, and we would like to find such anomalies.
In this case, this problem is unsupervised, we don't have data
where certain transactions are labelled as fraud that we can
go off.

\subsection{Exploratory Data Analysis}

As a starting point, the data is often not structured for use
in anomaly detection. First, the data can be structured into
well-understood variables and tables, and we can look at some
simple descriptives of the data: missing, invalid, extremes and
outlier values. For example for healthcare:

\begin{itemize}
  \item Large amounts of declarations on a single day
  \item Missing or invalid treatment codes
  \item Large amounts of treatments within a single day
\end{itemize}

This generally results in the formulation of rules, with the
input from domain experts:

\paragraph{Soft Rules}

We don't want to be too strict on detecting fraud, as there can
be a lot of simple mistakes, or errors that \textit{can} happen
but are unlikely. Rather, we can allow things but flag them when
they happen too frequently, i.e. above the 99th percentile.

\paragraph{Hard Rules}

Rules that are fully illegal and there should be no leeway for
using that rule. For example, receiving the same transaction twice,
booking the same consultation multiple times.

\subsection{Regression Model}

A case where we are able to train supervised models is to validate
if the quotes that were given for a (insured) repair is an anomaly.
First, we train a regression model on our input data, and use the
residual as an anomaly score. Given a new case, apply the model and
see if the residual is larger than some threshold (generally use the
$\sigma$).

\subsection{Local Outlier Factor}

Many algorithms to find outliers use the density, if it is
outside of the confidence intervals of a distribution, we consider
it an outlier or anomaly. However, the density is not always uniform
and points outside the distribution do not have to be outliers.
The idea behind the Local Outlier Factor is to use the density
of the neighbouring points, instead of just the considered point.

We compute the neighbours, and calculate what the distance is to
the $k$-th neigbour (and skip the $k-1$ neighbour). Compare the density
to the density of neighbours. If the distance is much larger, consider
it an outlier.

\begin{definition}[Local Outlier Factor]
  ~
  \begin{itemize}
    \item Compute the distance between two points $a$ and $b$: $d(a, b)$.
    \item Determine the $k$-distance of a point $a$, which is the
      distance to its $k$-th nearest neighbor.
    \item Calculate the reachability distance of a point $b$ with
      respect to a point $a$:
      \[
        \text{reach-dist}_k(a, b) = \max\big(\text{k-distance}(b), d(a, b)\big).
      \]
    \item Compute the average reachability distance of a point $a$:
      \[
        \text{avg-reach-dist}_k(a) = \frac{\sum_{b \in
          \text{neighbors}_k(a)} \text{reach-dist}_k(a,
        b)}{|\text{neighbors}_k(a)|}.
      \]
    \item Calculate the local reachability density of $a$:
      \[
        \text{lrd}_k(a) = \frac{1}{\text{avg-reach-dist}_k(a)}.
      \]
    \item Compute the Local Outlier Factor (LOF) score for $a$:
      \[
        \text{LOF}_k(a) = \frac{\sum_{b \in \text{neighbors}_k(a)}
        \text{lrd}_k(b)}{\text{lrd}_k(a) \cdot |\text{neighbors}_k(a)|}.
      \]
  \end{itemize}
\end{definition}

Advantages are that local density estimation works pretty well, is
well applicable
and there are efficient implementations available. However, the
scores are difficult
to interpret and don't have a clear boundary between inliers and
outliers. Moreover,
it doesn't work with high-dimensionality because
nearest-neighbourhoods algorithms
don't work with it. Difficult to extract or apply a dissimilarity feature.

\subsection{Isolation Trees}

The main idea is that we build a decision tree, and a shallow leaf will indicate
that it classifies an outlier. The intuition is that deep leafs were
difficult to
separate from other elements, and shallow leafs were obvious cases.
The approach is to train a random forest of the data and average the isolation
depth for attributes.

In this case, the anomaly score can be calculated from the expected path length
in a tree of $n$ nodes, where $H(i)$ is the $i$-th harmonic number,
defined as $ln(i + \gamma)$
and $\gamma = 0.577721$ (Euler-Mascheroni constant):

\begin{align*}
  c(n) = 2H(n - 1) - \frac{2(n - 1)}{n}
\end{align*}

The anomaly score for a point $x$ is then given by:

\begin{align*}
  s(x, n) = 2^{-\frac{h(x)}{c(n)}},
\end{align*}

Where $h(x)$ is the average path length of $x$ in the
isolation trees.

\begin{itemize}
  \item If $s(x, n) > 0.5$, the point $x$ is likely an outlier.
  \item If $s(x, n) \approx 0.5$, the point $x$ is likely an inlier.
  \item If $s(x, n) < 0.5$, the point $x$ is likely central.
\end{itemize}


\end{document}
