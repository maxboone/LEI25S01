\section{Random Forests and Ensembles}

Random Forests are supervised learning models that can be used for
both classification and regression tasks. They are based on an
ensemble of decision trees, which individually split the dataset into
subsets by evaluating feature values. These splits occur at multiple
levels, enabling the model to progressively refine its predictions
for the target variable.

\subsection{Ensembles}

An ensemble in supervised learning is a collection of models that
works by aggregating the individual predictions. Generally, it is more
accurate than the base model. Regression generally averages the indivdual
predictions, and classification uses a majority vote.

It helps if there is more diversity between models, which can be achieved by
using randomization or multiple types of classifier models.

\subsubsection{Bagging}

Bootstrap Aggregating, in short Bagging, is an early implementation of this
idea. Here each tree is bootstrapped with random samples with replacement
from the original dataset.

\begin{definition}[Bagging]
  \begin{itemize}
    \item Take random samples with replacement.
    \item Given a training set $D$ of size $n$, generate $m$
      new training sets $D_1, \dots, D_m$ each of size $n$.
    \item Replacement means that some observations will be
      repeated in each sample.
    \item For a large $n$, each $D_i$ will contain approximately $(1 -
      \frac{1}{e}) \approx 63.2\%$ unique samples and $36.8\%$ duplicates.
  \end{itemize}
\end{definition}

Overfitting is avoided in this case as the learners have little
correlation, given that they learn from different datasets. The optimal
number of learners can be determined by cross-validation or Out-of-Bag (OOB)
estimation.

\subsubsection{Random Subspace Method}

A follow-up of bagging is the random subspace method. Instead of
sampling, we build each tree in the ensemble from a random subset of
the attributes. This method is particularly effective for
high-dimensional problems
as individual trees are prevented from over-focusing on attributes
that appear most
predictive in the training set.

\subsubsection{Random Forests}

Random forests combine the ideas of bagging and the random subspace method.
At each split in a tree, a random subset of the attributes is
selected, and the best
split is chosen from this subset. The number of attributes selected is typically
$\sqrt{p}$ for classification and $\frac{p}{3}$ for regression, where
$p$ is the total
number of attributes. Random forests also employ out-of-bag (OOB) estimation for
model evaluation and tuning.

\begin{definition}[Random Forests]
  Random forests are an ensemble learning method that constructs
  multiple decision
  trees during training. Predictions are made by aggregating the
  outputs of individual
  trees (e.g., by majority voting for classification or averaging for
  regression).
  The randomness introduced in both the sampling of data and the selection of
  attributes ensures reduced overfitting and increased generalization.
\end{definition}

\begin{definition}[Out-of-Bag Estimation]
  Out-of-bag (OOB) estimation is a technique for evaluating the performance of
  ensemble models like random forests without the need for a separate
  validation set.
  During training, each tree is constructed using a bootstrap sample
  of the data, leaving
  approximately $36.8\%$ of the samples out of the bootstrap sample.
  These out-of-bag
  samples are used to estimate the model's performance by testing
  them on the trees
  that did not use them during training.
\end{definition}

The problem with Random Forests opposed to simple decision trees is
that they are not very transparent due to the splitting of the model
into multiple trees. To figure out the important attributes you can
record the average OOB-error over all trees: $e_0$, then over each
independent attribute $j$:

\begin{itemize}
  \item Shuffle the values of the attribute $j$, such that it only gives noise
  \item Refit the Random Forest
  \item Record the average OOB error $e_j$
  \item Importance of $j$ is the difference $e_j - e_0$
\end{itemize}

