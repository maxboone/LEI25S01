\section{Boosting}

Given a collection of learners in the decision tree, consider
a class of weak learners $h_m$ which is only slightly better
than random guessing. These are generally leafs of the decision
tree and are also called ``decision stumps''.

Build a sequence of classifiers $h_1, h_2, ..., h_k$:
\begin{itemize}
  \item $h_1$ is trained on the original data
  \item $h_2$ assigns more weight to misclassified cases by $h_1$
  \item $h_i$ assigns more weight to misclassified cases by $h_1, ..., h_{i-1}$
\end{itemize}

The final classifier is then an ensemble of $h1,...,h_k$ with
weights $\alpha_m$ per classifier $h_m$:
$G(x) = \text{sign}(\sum a_m h_m(x))$.

\begin{definition}[Learner Weights]
  The weight $\alpha_m$ for a learner is determined by its performance,
  quantified by the error rate $\epsilon_m$. Higher-performing learners
  receive greater positive weights, while poor-performing learners may
  receive negative weights. The weight is computed as:
  \begin{align*}
    \alpha_m = \frac{1}{2} \ln \left( \frac{1 - \epsilon_m}{\epsilon_m} \right),
  \end{align*}
  where $\epsilon_m$ is the error rate of learner $m$.
\end{definition}

\begin{definition}[Data Weights]
  On each iteration $G_m$ to $G_{m+1}$, we increase the weights
  for $G_{m+1}$ based on the $G_m$'s misclassified cases. We increase
  the weight by a factor $\exp(\alpha_m)$.
\end{definition}

\subsection{AdaBoost}

AdaBoost is an implementation of this idea, and works as follows:

\begin{definition}[AdaBoost]
  ~
  \begin{itemize}
    \item Initialize the observation weights $w_i = \frac{1}{N}$
    \item Loop over the classifiers $M$ as $m$
      \begin{itemize}
        \item Fit a classifier $h_m(x)$ to the training data using weights $w_i$
        \item Compute the weighted error $\epsilon_m = \frac{
            \sum w_i \times I(y_i \neq h_m(x_i))
          }{
            \sum w_i
          }$
        \item Compute $\alpha_m = \frac{1}{2} \ln ( \frac{1 -
          \epsilon_m}{\epsilon_m} )$
        \item For $i = 1$ to $N$
          \begin{itemize}
            \item If $x_i$ is misclassified, set data weight $w_i$ to
              $w_i \times \exp(\alpha_m)$
          \end{itemize}
      \end{itemize}
    \item Output $G(x) = \text{sign}(\sum \alpha_m h_m(x))$
  \end{itemize}
\end{definition}

\subsection{Gradient Boosting}

An alternative to AdaBoost is Gradient Boosting that has a bit more
theory behind the way how each iteration works. It uses a cost function
$L(u, F(x))$ which is the MSE in regression and log loss in the
classification. This loss function is differentiable and we can
use gradient descent here.

\begin{definition}[Gradient Boosting]
  ~
  \begin{itemize}
    \item Initialize the model with a constant value $F_0$, e.g. the mean
      of target values.
    \item For $m = 1$ to $M$
      \begin{itemize}
        \item Train weak learner $h_m(x)$ to minimize residuals from the current
          prediction using the loss function (c.q. train the model to
            the negative
          gradients).
        \item New prediction is $F_{m+1}(x) = F_m(x) + v \times h_m(x)$, where
          $v$ is the learning rate (e.g. 0.1)
      \end{itemize}
    \item Output $F_M(x)$ as the final model
  \end{itemize}
\end{definition}

\subsubsection{Loss Functions}

By default, we look at the MSE, but there are other loss functions to use for
the gradient descent as well, with the square loss repeated:

\begin{definition}[Square Loss]
  \begin{displaymath}
    L(y, F) = \frac{1}{2} (y - F)^2
  \end{displaymath}
\end{definition}

\begin{definition}[Absolute Loss]
  \begin{displaymath}
    L(y, F) = \left| y - F \right|
  \end{displaymath}
\end{definition}

\begin{definition}[Huber Loss]
  \begin{displaymath} L(y, F) =
    \begin{cases}
      \frac{1}{2} (y - F)^2 & \text{if } |y - F| \leq
      \delta \\
      \delta \left( |y - F| - \frac{1}{2} \delta \right) &
      \text{if } |y - F| > \delta
    \end{cases}
  \end{displaymath}

  Where $\delta$ refers to a parameter that decreases how
  extreme the output should be with respect to higher
  outliers.
\end{definition}

Avoiding overfitting can be done by lowering the learning rate
(in other words, shrinkage). Or, stochastic gradient boosting,
where each base learner is trained from a random sample of the
data with fraction $f$.

\subsubsection{XGBoost}

A modern and highly succesful version of Gradient Boosting,
that did some improvements such as handling of missing data,
L1 and L2 regularization that adds a penalty to loss functions
and it uses second-order derivatives for more accurate
optimization.
