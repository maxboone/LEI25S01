\documentclass[a4paper,10pt]{article}
\usepackage{graphicx}
\usepackage{amsmath}
\usepackage{hyperref}
\usepackage{listings}
\usepackage{geometry}
\geometry{margin=1in}

\title{
  Branch-prediction effects in quick sort \\
\small QuickSort with and without branching in the partitioning function}
\author{Max Boone (s2081318)}
\date{\today}

\begin{document}

\maketitle

\section{Introduction}

This report briefly describes and benchmarks two implementations of
quicksort, one implementation that uses an branching if-statement in
its partitioning function and one that uses an assigned conditional
to avoid branch misses. First, the implementation of the sorting
algorithm is discussed. Then, performance results of sorting are
provided.

\section{Implementation}

The quicksort algorithm works in two main steps. First, a partition
function divides the list into two parts by swapping elements to
ensure those smaller than a chosen pivot are on the left and larger
ones are on the right. Second, quicksort is applied recursively to
each part until the whole list is sorted.

In the partitioning function, a conditional swapping is done based on
if the element is smaller than the chosen pivot. Due to the randomness
of this operation the branch predictor is likely to mispredict whether
to jump or not to the swapping instruction. As an optimization, this is
replaced by a conditional assignment and a temporary variable, to avoid
having to jump to another instruction.

\section{Performance}



\section{Conclusion}

\end{document}
