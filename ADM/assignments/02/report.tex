\documentclass[a4paper,10pt]{article}
\usepackage{graphicx}
\usepackage{amsmath}
\usepackage{hyperref}
\usepackage{listings}
\usepackage{geometry}
\geometry{margin=1in}

\title{
  NYC Yellow Taxi Dataset
\small Performance Comparison of Pandas, SQLite, and DuckDB}
\author{Max Boone <s2081318>}
\date{\today}

\begin{document}

\maketitle

\section{Introduction}

This report provides a short analysis of the performance differences
between processing and analyzing the NYC Cab dataset from 2016. All
analyses are done through Pandas, SQLite and DuckDB. Experiments are 
executed on a CAX41 machine from Hetzner Cloud. This
machine has 32 GiB ECC RAM and 16 Neoverse N1 cores.

Teh report is structured as follows. First, the data
loading step is reviewed. Second, two queries that test general
performance of the database are run and discussed. Finally, a simple
linear model on the data is calculated and discussed.

\section{Data Loading}

The dataset was provided as a parquet file. Pandas and DuckDB offer
native functionality to load parquet files into the internal data
structure. SQLite does not offer this functionality it self, and
the data is first loaded into memory by pandas before writing to
SQLite. Only the SQLite part is measured there. 

The execution times are as follows:

\begin{itemize}
  \item \textbf{Pandas:} 1.30 seconds
  \item \textbf{SQLite:} 109.71 seconds
  \item \textbf{DuckDB:} 12.70 seconds
\end{itemize}

DuckDB and SQLite store the resulting database file on disk and this
already shows some difference in performance. The resulting SQLite database
is around 1.2 GiB in size and the DuckDB database around 211 MiB. For a fairer
comparison, and to rule out disk overhead, a cursor for SQLite and DuckDB was
also set up in-memory.

\begin{itemize}
  \item \textbf{SQLite (Memory):} 106.50 seconds
  \item \textbf{DuckDB (Memory):} 5.97 seconds
\end{itemize}

One of the reasons that SQLite is this slow in data loading might be that pandas
presents the entire dataset in one query to the SQLite engine instead of in batches
that can be written asynchronously to the WAL.

\section{Query Performance}
\subsection{Query 1: \textit{[Distinct Counts]}}

The first query counts the distinct values for each dataset, the execution times
are as follows:

\begin{itemize}
  \item \textbf{Pandas}: 2.80s
  \item \textbf{SQLite}: 37.34s
  \item \textbf{SQLite (Memory)}: 37.79s
  \item \textbf{DuckDB}: 0.53s
  \item \textbf{DuckDB (Memory)}: 0.47s
\end{itemize}

DuckDB is a clear winner in this section, as is expected from the columnar storage
format of the database. Likely the strategy that it uses to count these columns is
as efficient as Pandas, but it can partition the query and use multiprocessing. SQLite
likely retrieves every row for every column which causes a lot of duplicate values to
be loaded. It is however strange that the in-memory database does not perform better,
possibly due to the amount of available memory the caches are doing their work well.

\subsection{Query 2: \textit{[Aggergation and Grouping]}}

The second query aggregates data from the dataset and gives descriptives for each
day-hour combination. The execution times are as follows:

\begin{itemize}
  \item \textbf{Pandas}: 0.93s
  \item \textbf{SQLite}: 13.51s
  \item \textbf{SQLite (Memory)}: 13.26s
  \item \textbf{DuckDB}: 0.03s
  \item \textbf{DuckDB (Memory)}: 0.03s
\end{itemize}

Again the queries from DuckDB are the fastest, likely this is due to the implementation
of the fully parallelized aggregate hash table that DuckDB employs. Other time differences
are likely due to the same reasons as discussed in the previous query.

\section{Machine Learning: Fare Estimation}

\begin{itemize}
  \item \textbf{Pandas:} GG seconds
  \item \textbf{SQLite:} HH seconds
  \item \textbf{DuckDB:} II seconds
\end{itemize}
\textbf{Discussion:} Analysis of why certain systems performed better
or worse for machine learning tasks, considering factors such as
computational overhead, internal optimizations, and memory management.

\section{Conclusion}
This report has provided a comparison of Pandas, SQLite, and DuckDB
in terms of their performance for loading data, querying, and
performing machine learning tasks. Based on the observed execution
times, DuckDB shows advantages in XYZ tasks, while Pandas/SQLite
performed better in ABC tasks. Further research could investigate
additional optimizations and the scalability of these systems with
larger datasets.

\end{document}
